\section{File Formats\label{s:formats}}

\subsection*{Vector}

A vector file begins with a single positive integer,
which specifies the number of elements in the vector.
This is then followed by N rows,
each containing a single value.

\subsection*{Matrix}

A file containing a matrix begins with a pair of positive integers,
which specify the number of rows and columns in the matrix respectively.
(Note that this means the first number is the Y extent, and the second number is the X extent.)
Elements of the vector or matrix then appear one per line in order of increasing index,
i.e.,
the element at (1,1) appears first,
then the element at (1,2),
and so on up to (1,N),
which is followed by the element at index (2,1).

\subsection*{Basic Types}

Vectors and matrices may contain Booleans, integers, or reals;
vectors may also contain (x,y) points.
The two Boolean values are represented by upper-case 'T' and 'F'.
Integers and reals are represented in the usual way;
points are represented as two numbers separated by a single space character.
